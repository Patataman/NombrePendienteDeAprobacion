\documentclass[10pt,a4paper,titlepage]{article}
\usepackage[utf8]{inputenc}
\usepackage[spanish]{babel}

\begin{document}

\section{PalabraMaldita básicos}
\begin{itemize}
	\item Resolución fija, 1024×768 de base.
	\item La fuente que se utilizará será Lato o Montserrat.
	\item Menú selección personajes.
	\item Música. Sin opción a quitarla.
	\item Dos personajes base más los de internete (por eso es Dimensions (\textbf{Pendiente de aprobación})).
	\item Personajes pueden saltar. El salto es del doble de la altura del personaje.
	\item De primeras dos ataques básicos: Normal y Fuerte.
	\item En el aire también se puede pegar pero 1 sólo golpe, independientemente del botón de pegar.
	\item En el suelo:
	\begin{itemize}
		\item Estar quieto: Mismo ataque que moverse en la dirección a la que está el oponente.
		\item Dirección atrás: Se hace un ataque distinto. Depende del botón.
		\item Dirección hacia arriba ==== SALTAR
		\item Dirección abajo: Se hace un golpe único, similar a saltar, pero DEPENDE del botón.
		\item Moverse hacia atrás sin atacar === CUBRIRSE
		\item NO HAY SECUENCIA DE ATAQUES. LOS GOLPES SON SUELTOS.
		\item \textbf{PENDIENTE DE APROBACIÓN}: Pulsar los 2 botones de ataque hace ataque especial (JOYIUKEN).
	\end{itemize}

	\item Barras de vida: Los personajes tienen 100\% de vida siempre. La barra de vida baja en función de:
	\begin{enumerate}
		\item Golpes adelante y atrás
		\begin{itemize}
			\item Golpes flojos: Quitan 5. 
			\item Golpes fuertes: Quitan 10.
		\end{itemize}
		\item Golpes saltando
		\begin{itemize}
			\item Quitan 3.
		\end{itemize} 
		\item Hacia abajo.
		\begin{itemize}
			\item Golpes flojos: Quitan 3. 
			\item Golpes fuertes: Quitan 6.
		\end{itemize}
	\end{enumerate}

	\item Como máximo se pueden repetir 3 veces un golpe flojo. En el suelo.
	\item Como máximo sólo se puede realizar un golpe fuerte.
	\item Tiempo entre cadenas 1s (PROVISIONAL).

	\item Velocidad de los ataques
	\begin{enumerate}
		\item Golpes adelante y atrás
		\begin{itemize}
			\item Golpes flojos: 1/3 de segundo (PROVISIONAL). 
			\item Golpes fuertes: 1/2 segundo (PROVISIONAL).
		\end{itemize}
		\item Golpes saltando: 1/3 de segundo (PROVISIONAL).

		\item Hacia abajo.
		\begin{itemize}
			\item Golpes flojos: 1/3 de segundo (PROVISIONAL). 
			\item Golpes fuertes: 1/2 segundo (PROVISIONAL).
		\end{itemize}
	\end{enumerate}

	\item Tiempo de combate: Los combates durarán 1 minuto (60s) como máximo (PROVISIONAL).
	\item Habrá que ganar 2 veces para ser vencedor del combate (Al mejor de 3).
	\item Habrá una cuenta atrás de 3 segundos antes de cada combate/ronda.
	\item Los personajes se resetean a la posición inicial cada ronda.
	\item El juego tendrá un menú de pausa (Tecla ESCAPE). En este menú se podrá: REANUDAR COMBATE, SALIR A SELECCIÓN y SALIR DEL JUEGO.
	\item Los personajes tendrán al menos un icono de personaje, para la selección y para colocarlo al lado de la barra de vida.
	\item SONIDOS.
	\begin{itemize}
		\item Todos los golpes tendrán al menos 1 sonido (Puede ser el mismo).
		\item Cuando un personaje recibe un golpe debe hacer un sonido (AAARRRGGG).
	\end{itemize}

	\item Habrá un fondo de pantalla. Ya veremos como.
\end{itemize}

Definición básica de clases
\begin{itemize}
	\item Clase personaje: Clase padre de la que derivarán todos los demás personajes.
	\begin{itemize}
		\item Nombre: Se explica solo 
		\item Sprites[] (Lista): Cada posición de la lista se corresponde con una tecla de acción válida. De esta forma se `adapta" para futuros combos.
				TODAS LAS FICHAS DE SPRITES DEBERÁN LLAMARSE DE LA SIGUIENTE FORMA: <nombre>_ficha.png
		\begin{itemize}
			\item <nombre>: Nombre del personaje, todo en minúsculas.
		\end{itemize}
		\item Estado: Variable numérica que representará el estado del personaje
		\begin{itemize}
			\item Estado idle: Valor 0
			\item Estado atacando: Valor 1
			\item Estado defendiendo: Valor 2
			\item Estado en el aire: Valor 3
			\item Estado atacando en el aire: Valor 13
			\item Estado moviendose hacia atras en el aire: Valor 23
			\item Estado recibiendo damages: Valor 4
			\item Estado orientacion: Valor `izq" o `drch", indica hacia que lado está el oponente
		\end{itemize}
		\item Vida: Valor numérico de 0 a 100 de tipo entero.
	\end{itemize}
\end{itemize}

Definicion de animaciones 
\begin{itemize}
	\item Avatar del personaje: Para el panel de personajes y para colocarlo al lado de la barra de vida.
	\item Animación de ataque fuerte, ataque débil, idle, morir, saltar, pegar saltando, golpe bajo, defenderse y defenderse en el aire: Mínimo 10 frames por cada animación. (Véase hoja de Samus, se puede negociar el nº mínimo de frames).
	\begin{itemize}
		\item Ataque fuerte o débil: Se realiza cuando se pulsa la tecla correspondiente
		\item Idle: Animación que se realiza cuando el personaje está quieto sin recibir daño.
		\item Saltar: Animacion que se realiza cuando el pj salta.
		\item Pegar saltando: Animación que se realiza cuando el personaje realiza un golpe en medio de un salto.
		\item Golpe bajo: Animación que se realiza cuando el personaje realiza un ataque en dirección hacia abajo (RECORDATORIO: El personaje no se puede agachar)
		\item Defenderse y defenderse en el aire: Animación (2 o 3 frames a lo muchisimo xddd) que se realiza cuando el personaje se defiende.
	\end{itemize}
	\item Sprites[] (Lista): Cada posición de la lista se corresponde con una tecla de acción válida. De esta forma se `adapta" para futuros combos.
	\item Los sprites deberán ser a color.
	\item Resolución de los frames: PENDIENTE DE CONCRETAR.
	\item OPCIONALES: 
	\begin{itemize}
		\item Imagen de avatar `Recibiendo daño".
		\item Animación barra de vida.
	\end{itemize}
\end{itemize}


TAREAS PARA LA SEMANA PRÓXIMA

-- CREAR CLASE PERSONAJE (ADRI)
-- CREAR MENU PRINCIPAL (DANI)
-- CREAR PANEL SELECCION (DANI)
-- CREAR ESCENA PELEAS (DANI)
-- CREAR snippet DE CÓDIGO PARA EL HUD DE VIDA (ADRI)

\end{document}
